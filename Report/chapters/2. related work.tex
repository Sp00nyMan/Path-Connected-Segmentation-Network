\section{Background and Related Work}

The field of image segmentation has been a subject of extensive research over the years,
with deep learning techniques playing a pivotal role.
The use of deep learning in image segmentation has been explored in numerous studies, such as the work by \citet{ronneberger2015u} on U-Net,
a convolutional network for biomedical image segmentation.
This work demonstrated the effectiveness of deep learning in capturing intricate patterns and producing high-quality segmentation masks.
However, the vulnerability of deep learning models to adversarial attacks and noise has been a persistent challenge,
as highlighted by \citet{szegedy2013intriguing}.

The concept of adversarial attacks was first introduced by \citet{szegedy2013intriguing}, where they demonstrated that adding imperceptible perturbations
to the input can lead to misclassification by deep neural networks.
This vulnerability has been a significant concern in the field of image segmentation, as it can lead to inaccurate segmentation masks.
\citet{goodfellow2014explaining} further explored this issue, proposing a family of fasts methods for generating adversarial examples
and suggesting potential countermeasures.

To address these vulnerabilities, researchers have proposed various techniques to improve the robustness of deep learning models.
For instance, \citet{madry2017towards} proposed a robust optimization framework for training models that are resilient to adversarial attacks.
However, these methods often require significant computational resources and do not guarantee the path connectedness of the decision space.

In the realm of image segmentation, Normalizing Flows and Input-Convex Neural Networks have been used to improve the robustness and
interpretability of models. Normalizing Flows, as popularized by \citet{rezende2015variational} and \citet{dinh2016density},
provide a framework for constructing complex distributions by transforming a simple initial distribution. Input-Convex Neural Networks, on the other hand, were introduced by \citet{amos2017input}.
They demonstrated that these networks can enforce certain mathematical properties on the model, leading to more interpretable results.
However, their work did not specifically address the issue of path connectedness in the decision space.

In this paper, we aim to bridge this gap by introducing a technique that guarantees the path connectedness of the decision space in image segmentation
models, thereby improving their robustness and interpretability.